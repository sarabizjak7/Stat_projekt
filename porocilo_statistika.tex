\documentclass[a4paper,12pt]{article}

\usepackage[slovene]{babel}
\usepackage[utf8]{inputenc}
\usepackage[T1]{fontenc}
\usepackage{lmodern}
\usepackage{amsmath}
\usepackage{amssymb}
\usepackage{amsthm}
\usepackage{amsfonts}
\usepackage{mathtools}
\usepackage{enumitem}



\begin{document}

\begin{titlepage}
    \centering
    \vfill
    \vfill
    \textbf{\large{POROČILO SEMINARSKE NALOGE}}
    \vfill
    \textbf{\Huge{STATISTIKA}}
    \vfill\vfill\vfill\vfill\vfill
    \textsc{\Large{Sara Bizjak}}
    \vfill\vfill
    \textsc{\large{Univerza v Ljubljani}}
    
    \textsc{\large{Fakulteta za matematiko in fiziko}}
    
    \textsc{\large{Oddelek za matematiko}}
    \vfill\vfill
        
    \large{JULIJ 2020}
    
\end{titlepage}

%%%

\newpage

%%%%%%%%%%%%%%%%%%%%%%%%%%%%%%%%%%%%%%%%%%%%%%%%%%%%%%%%%%%%%%

\noindent
\textsc{\large{1. NALOGA}}
\\
\\
Podatki so vzeti iz datoteke \texttt{Kibergard}, kjer se nahajajo informacije o 43.886 družinah, ki stanujejo v mestu Kibergard. Za vsako družino so zabeleženi naslednji podatki:
\begin{itemize}
    \item Tip družine (od 1 do 3)
    \item Število članov družine
    \item Število otrok v družini
    \item Skupni dohodek družine
    \item Mestna četrt, v kateri stanuje družina (od 1 do 4)
    \item Stopnja izobrazbe in vodje gospodinjstva (od 31 do 46: opisi v datoteki z navodili)
\end{itemize}

\noindent
Nalogo sem reševala s pomočjo programa R. Koda, uporabljena za generiranje enostavnih slučajnih vzorcev in izračune, je dostopna v priloženi datoteki \textit{naloga1.R}. 
\\

%%%%%%%%%%%%%%%%%%%%%%%%%%%%%%%%

\noindent
\textsc{Primer a}
\\
Vzamemo enostavni slučajni vzorec 200 družin in na njegovi podlagi ocenimo delež družin v Kibergardu, v katerih vodja gospodinjstva nima srednješolske izobrazbe (niti poklicne niti splošne mature).
Opisan delež znaša $p = 0.195$.
\\

%%%%%%%%%%%%%%%%%%%%%%%%%%%%%%%%

\noindent
\textsc{Primer b}
\\
Ocenimo standardno napako in postavimo 95\% interval zaupanja.

\noindent
Standardno napako za delež izračunamo po formuli
$$ \hat{se}(p) = \sqrt{ \frac{p \cdot (1-p)}{n - 1} \cdot \left(1 - \frac{n}{N} \right)}, $$
kjer so $p = 0.195$, \ $n = 200$, \ $N = 43.886$. 
\\
Dobimo rezultat $\hat{se}(p) = 0.02802185$.
\\
Interval zaupanja je enak: $[0.1400782, 0.2499218]$.
\\

%%%%%%%%%%%%%%%%%%%%%%%%%%%%%%%%

\noindent
\textsc{Primer c}
\\
Vzorčni delež in ocenjeno standardno napako primerjamo s populacijskim deležem in pravo standardno napako. 
\\
\begin{itemize}
\item Vzorčni delež: $0.195$ 
\item Populacijski delež: $0.2115025$ 
\item Razlika obeh deležev: $0.01650253$
\item Ocenjena standardna napaka (iz vzorca): $0,02802185$
\item Prava standardna napaka (iz celotne populacije):  $0.02888282$
\item Razlika med ocenjeno in pravo standardno napako: $0.0008609634$
\end{itemize}
Ker velja $0.2115025 \in [0.1400782, 0.2499218]$, interval zaupanja pokrije populacijski delež.
\\

%%%%%%%%%%%%%%%%%%%%%%%%%%%%%%%%

\noindent
\textsc{Primer d}
\\
GRAF 
\\
INTERVALI ZAUPANJA + KOLIKO JIH POKRIJE POPULACIJSKI DELEŽ
\\

%%%%%%%%%%%%%%%%%%%%%%%%%%%%%%%%

\noindent
\textsc{Primer e}
\\
Standardni odklon vzorčnih deležev za $100$ prej dobljenih vzorcev je enak $0.02881085$. Prava standardna napaka za vzorec velikosti $200$ pa je $0.02888282$.
Razlikujeta se za $7.196778 \cdot 10^{-5}$.
\\

%%%%%%%%%%%%%%%%%%%%%%%%%%%%%%%%

\noindent
\textsc{Primer f}
\\
GRAF 
\\
INTERVALI ZAUPANJA + KOLIKO JIH POKRIJE POPULACIJSKI DELEŽ

%%%%%%%%%%%%%%%%%%%%%%%%%%%%%%%%%%%%%%%%%%%%%%%%%%%%%%%%%%%%%%

\noindent
\textsc{\large{2. NALOGA}}

%%%%%%%%%%%%%%%%%%%%%%%%%%%%%%%%%%%%%%%%%%%%%%%%%%%%%%%%%%%%%%

\noindent
\textsc{\large{3. NALOGA}}
\\
\\
Opazimo $n$ neodvisnih realizacij zvezne porazdelitve z gostoto:
\begin{displaymath}
    f(x \ | \ \sigma) = \left\{ \begin{array}{ll}
     \frac{\Gamma(2 \alpha)}{(\Gamma (\alpha))^2} \left[ x(1 - x) \right]^{\alpha - 1} & \textrm{; \ $0 < x < 1$}\\
     0 & \textrm{; \ sicer,} \\
    \end{array} \right. 
\end{displaymath}
kjer je $\alpha > 0$ neznam parameter. Če je $X$ slučajna spremenljivka s to gostoto, se da izračunati:
$$ \text{E}(X) = \frac{1}{2}, \ \ \ \ \ \text{var}(X) = \frac{1}{4(2 \alpha + 1)}. $$
\\

\noindent
\textsc{Primer a}
\\
Določimo obliko porazdelitve v odvisnosti od $\alpha$.
\\

\noindent
\textsc{Primer b}
\\
Ocenimo $\alpha$ po metodi momentov.
\\

\noindent
\textsc{Primer c}
\\
Poiščemo enačbo, ki določa cenilko po metodi največjega verjetja. Pogledamo, kdaj ta cenilka obstaja.
\\

\noindent
\textsc{Primer d}
\\
Poiščemo asimptotično varianco cenilke po metodi največjega verjetja.
\\
 
%%%%%%%%%%%%%%%%%%%%%%%%%%%%%%%%%%%%%%%%%%%%%%%%%%%%%%%%%%%%%%

\noindent
\textsc{\large{4. NALOGA}}

%%%%%%%%%%%%%%%%%%%%%%%%%%%%%%%%%%%%%%%%%%%%%%%%%%%%%%%%%%%%%%

\noindent
\textsc{\large{5. NALOGA}}
\\
\\
$X$ in $Y$ sta slučajni spremenljivki, za kateri velja:
\begin{itemize} 
    \item $\text{E}(X) = \mu_x$,
    \item $\text{E}(Y) = \mu_y$,
    \item $\text{var}(X) = \sigma_x^2$,
    \item $\text{var}(Y) = \sigma_y^2$,
    \item $\text{cov}(X,Y) = \sigma_{x,y}$.
\end{itemize}
Opazimo $X$ in želimo napovedati $Y$. 
\\

\noindent
\textsc{Primer a}
\\
Poiščemo napoved, ki je oblike $\hat{Y} = \alpha + \beta \cdot X$, kjer $\alpha$ in $\beta$ izberemo tako, da je srednja kvadratična napaka $\text{E} \left[ \left( Y - \hat{Y} \right) ^2 \right]$ minimalna.
\\
Uporabimo namig 
\begin{align} 
    \text{E} \left[ \left( Y - \hat{Y} \right) ^2 \right] = \left[ \text{E}(Y) - \text{E}(\hat{Y}) \right]^2 + \text{var}(Y - \hat{Y}).
\end{align}
Ker sta oba člena desne strani enačbe večja od $0$, je dovolj, da poiščemo vrednosti $\alpha$ in $\beta$, ki minimizirata ta dva člena -- potem bo najmanjša možna tudi njuna vsota.
\\
Poglejmo si najprej prvi člen enačbe. Vrednost enačbe

\begin{align*}
    \left[ \text{E}(Y) - \text{E}(\hat{Y}) \right]^2 = \left[ \mu_y - \alpha - \beta \mu_x \right] ^2 
\end{align*}
bo najmanjša, ko bo $ \mu_y - \alpha - \beta \cdot \mu_x = 0$. To pa bo res natanko takrat, ko bo $\alpha = \mu_y - \beta \mu_x$. 
\\
Poglejmo si še drugi člen enačbe. Vidimo, da je 

\begin{align*} 
\text{var}(Y-\hat{Y}) &= \text{var}(Y - \alpha - \beta X) = \text{var}(Y - \beta X) = \text{var}(Y) - 2 \beta \text{cov}(X, Y) + \beta^2 \text{var}(X)
\\
&= \sigma_y^2 - 2 \beta \sigma_{x,y} + \beta^2 \sigma_x^2 
\end{align*}
funkcija spremenljivke $\beta$. Za izračun minimuma enačbo najprej odvajamo po $\beta$ in enačimo z $0$.

\begin{align*}
    \frac{\partial}{\partial \beta} (\text{var}(Y - \hat{Y})) = - 2 \sigma_{x,y} + 2 \beta \sigma_x^2 = 0 
\end{align*}
To bo res, ko bo $ \beta = \frac{\sigma_{x,y}}{\sigma_x^2} $.
\\
Vrednosti $\alpha$ in $\beta$, pri katerih bo izraz $\text{E} \left[ \left( Y - \hat{Y}) \right) ^2 \right]$ minimalen, sta
$$ \alpha = \mu_y - \mu_x \frac{\sigma_{x,y}}{\sigma_x^2} \ \ \ \ \ \text{in} \ \ \ \ \ \beta = \frac{\sigma_{x,y}}{\sigma_x^2} $$

\noindent
\textsc{Primer b}
\\
Pokažemo, da se pri tako izbranih koeficientih \textit{determinacijski koeficient} (kvadrat korelacijskega koeficienta) izraža v obliki
$$ r_{x,y}^2 = 1 - \frac{\text{var}(Y - \hat{Y})}{\text{var}(Y)}. $$
\\
Spomnimo se najprej formule za \textit{korelacijski koeficient}: 
$$ \text{corr}(X,Y) = \frac{\text{cov}(X,Y)}{\sqrt{\text{var}(X) \text{var}(Y)}} = r_{x,y}. $$
Ker je determinacijski koeficient enak kvadratu korelacije, je enak
$$ r_{x,y}^2 = \frac{\text{cov}(X,Y)^2}{\text{var}(X) \text{var}(Y)}. $$
Dobimo enakost
$$ 1 - \frac{\text{var}(Y - \hat{Y})}{\text{var}(Y)} = \frac{\text{cov}(X,Y)^2}{\text{var}(X) \text{var}(Y)} $$
in če upoštevamo še vrednosti $\alpha$ in $\beta$ iz prvega dela naloge:
\begin{align*} 
    1 - \frac{\text{var}(Y - \hat{Y})}{\text{var}(Y)} &= \frac{\text{var}(Y) - \text{var}(Y - \hat{Y})}{\text{var}(Y)} = \frac{\sigma_y^2 - \sigma_y^2 + \frac{\sigma_{x,y}^2}{\sigma_x^2}}{\sigma_y^2} 
    &= \frac{\sigma_{x,y}^2}{\sigma_x^2 \sigma_y^2} = \frac{\text{cov}(X,Y)^2}{\text{var}(X) \text{var}(Y)} = r_{x,y}^2. 
\end{align*}
Pokazali smo, da je determinacijski koeficient res enak $1 - \frac{\text{var}(Y - \hat{Y})}{\text{var}(Y)}$.
%%%%%%%%%%%%%%%%%%%%%%%%%%%%%%%%%%%%%%%%%%%%%%%%%%%%%%%%%%%%%%
\end{document}
