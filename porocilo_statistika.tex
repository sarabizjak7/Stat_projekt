\documentclass[a4paper,12pt]{article}

\usepackage[slovene]{babel}
\usepackage[utf8]{inputenc}
\usepackage[T1]{fontenc}
\usepackage{lmodern}
\usepackage{amsmath}
\usepackage{amssymb}
\usepackage{amsthm}
\usepackage{amsfonts}
\usepackage{mathtools}
\usepackage{enumitem}



\begin{document}

\begin{titlepage}
    \centering
    \vfill
    \vfill
    \textbf{\large{POROČILO SEMINARSKE NALOGE}}
    \vfill
    \textbf{\Huge{STATISTIKA}}
    \vfill\vfill\vfill\vfill\vfill
    \textsc{\Large{Sara Bizjak}}
    \vfill\vfill
    \textsc{\large{Univerza v Ljubljani}}
    
    \textsc{\large{Fakulteta za matematiko in fiziko}}
    
    \textsc{\large{Oddelek za matematiko}}
    \vfill\vfill
        
    \large{JULIJ 2020}
    
\end{titlepage}

%%%

\newpage

%%%%%%%%%%%%%%%%%%%%%%%%%%%%%%%%%%%%%%%%%%%%%%%%%%%%%%%%%%%%%%

\noindent
\textsc{\large{1. NALOGA}}
\\
\\
Podatki so vzeti iz datoteke \texttt{Kibergard}, kjer se nahajajo informacije o 43.886 družinah, ki stanujejo v mestu Kibergard. Za vsako družino so zabeleženi naslednji podatki:
\begin{itemize}
    \item Tip družine (od 1 do 3)
    \item Število članov družine
    \item Število otrok v družini
    \item Skupni dohodek družine
    \item Mestna četrt, v kateri stanuje družina (od 1 do 4)
    \item Stopnja izobrazbe in vodje gospodinjstva (od 31 do 46: opisi v datoteki z navodili)
\end{itemize}

\noindent
Nalogo sem reševala s pomočjo programa R. Koda, uporabljena za generiranje enostavnih slučajnih vzorcev in izračune, je dostopna v priloženi datoteki \textit{naloga1.R}. 
\\

%%%%%%%%%%%%%%%%%%%%%%%%%%%%%%%%

\noindent
\textsc{Primer a}
\\
Vzamemo enostavni slučajni vzorec 200 družin in na njegovi podlagi ocenimo delež družin v Kibergardu, v katerih vodja gospodinjstva nima srednješolske izobrazbe (niti poklicne niti splošne mature).
Opisan delež znaša $p = 0.195$.
\\

%%%%%%%%%%%%%%%%%%%%%%%%%%%%%%%%

\noindent
\textsc{Primer b}
\\
Ocenimo standardno napako in postavimo 95\% interval zaupanja.

\noindent
Standardno napako za delež izračunamo po formuli (na strani 210 v knjigi \textit{John A. Rice: Mathematical Statistics and Data Analysis}):
$$ \hat{se}(p) = \sqrt{ \frac{p \cdot (1-p)}{n} \cdot \left(1 - \frac{n - 1}{N - 1} \right)}, $$
kjer so $p = 0.195$, \ $n = 200$, \ $N = 43.886$. 
\\
Dobimo rezultat $\hat{se}(p) = 0.02795203$.
\\
Interval zaupanja je enak: $[0.140215, 0.249785]$.
\\

%%%%%%%%%%%%%%%%%%%%%%%%%%%%%%%%

\noindent
\textsc{Primer c}
\\
Vzorčni delež in ocenjeno standardno napako primerjamo s populacijskim deležem in pravo standardno napako. 
\\
\begin{itemize}
\item Vzorčni delež: $0.195$ 
\item Populacijski delež: $0.2115025$ 
\item Razlika obeh deležev: $0.01650253$
\item Ocenjena standardna napaka (iz vzorca): $0,02802185$
\item Prava standardna napaka (iz celotne populacije):  $0.02881085$
\item Razlika med ocenjeno in pravo standardno napako: $0.0008609634$
\end{itemize}
Ker velja $0.2115025 \in [0.140215, 0.249785]$, interval zaupanja pokrije populacijski delež.
\\

%%%%%%%%%%%%%%%%%%%%%%%%%%%%%%%%

\noindent
\textsc{Primer d}
\\
GRAF 
\\
INTERVALI ZAUPANJA + KOLIKO JIH POKRIJE POPULACIJSKI DELEŽ
\\

%%%%%%%%%%%%%%%%%%%%%%%%%%%%%%%%

\noindent
\textsc{Primer e}
\\
Standardni odklon vzorčnih deležev za $100$ prej dobljenih vzorcev je enak $0.02881085$. Prava standardna napaka za vzorec velikosti $200$ pa je $0.02888282$.
Razlikujeta se za $7.196778 \cdot 10^{-5}$.
\\

%%%%%%%%%%%%%%%%%%%%%%%%%%%%%%%%

\noindent
\textsc{Primer f}
\\
GRAF 
\\
INTERVALI ZAUPANJA + KOLIKO JIH POKRIJE POPULACIJSKI DELEŽ
\\

%%%%%%%%%%%%%%%%%%%%%%%%%%%%%%%%%%%%%%%%%%%%%%%%%%%%%%%%%%%%%%

\noindent
\textsc{\large{2. NALOGA}}
\\
\\
Sklepi in izpeljave, ki jih bom uporabila pri vseh podnalogah, sledijo dokazu izreka v knjigi \textit{John A. Rice: Mathematical Statistics and Data Analysis} (stran $232$, $233$).
\\
Naredimo raziskavo na populaciji, ki ima $K$ stratumov z velikostmi $N_1, N_2, \ldots, N_K$. Denimo, da lahko izberemo vzorec velikosti $n$.
\\

\noindent
\textsc{Primer a}
\\
Denimo, da so stroški raziskave enaki $C = C_0 + n C_1$, kjer je $n$ število enot v vzorcu ($C_0$ je začetni stršek, $C_1$ pa je nadaljnji strošek na enoto). Pri danih sredstvih za raziskavo v višini $C$ poiščemo velikosti podvzorcev $n_1, n_2, \ldots, n_K$, pri katerih je varianca standardne cenilke populacijskega povprečja minimalna.
\\
Označimo z $\overline{X}$ populacijsko povprečje.
\\
Če imamo stratume velikosti $N_1, N_2, \ldots, N_K$, se moramo odločiti, kako velike vzorce bomo izbrali iz posameznih stratumov. Izbiramo tako, da bo standardna napaka cenilke $\overline{X}$ čim manjša. 
Poiščemo torej take $n_1, n_2, \ldots, n_K$, kjer $n = n_1 + n_2 + \ldots + n_K$, da bo $$ \text{var}(\overline{X}) = \sum_{i = k}^{K} W_i ^ 2 \left( \frac{ \sigma_k^2}{n_k} \right) \left( \frac{N_k - n_k}{N_k - 1} \right) $$
čim manjša.
Ker so v večini praktičnih situacij korekturni faktorji $\frac{N_k - n_k}{N_k - 1} \approx 1$, jih lahko zanemarimo. Rešujemo torej problem vezanega ekstrema:
$$ f(n_1, n_2, \ldots, n_K) = \sum_{k = 1}^{K} \frac{\sigma_k^2}{n_k} W_k^2 $$
$$ \text{z vezjo} \  C = C_0 + nC_1.$$
Sestavimo Lagrangeovo funkcijo:
$$ F(n_1, n_2, \ldots, n_K, \lambda) = f(n_1, n_2, \ldots, n_K) + \lambda (C_0 + \sum_{k = 1}^{K} n_k C_1 - C). $$
Zapišemo parcialne odvode funkcije $F$ in jih enačimo z $0$.
$$ \frac{ \partial F}{\partial n_i} = - \frac{ \sigma_i^2}{n_i^2} \  W_i^2 + \lambda \ C_1 = 0 \ \ \ \text{za} \ i = 1, 2, \ldots, K. $$
Izrazimo $n_i$ in dobimo sistem enačb
\begin{align}\label{en:nk}
n_i = \frac{W_k \sigma_i}{ \sqrt{ \lambda \ C_1}}
\end{align}
Da določimo $\lambda$, naredimo vsoto enačbe (\ref{en:nk}) po $k$, $k = 1, 2, \ldots, K.$
$$ n = \frac{1}{\sqrt{\lambda \ C_1}} \sum_{k = 1}^{K} W_k \sigma_k $$
\begin{align}\label{en:lambda}
\Rightarrow \sqrt{\lambda} = \frac{\sum_{k = 1}^{K} W_k \sigma_k}{n \sqrt{C_1}}
\end{align}
Če združimo (\ref{en:nk}) in (\ref{en:lambda}), dobimo:

$$ n_i = n \ \frac{W_i \sigma_i}{\sum_{k = 1}^{K} W_k \sigma_k}. $$
Rezultat je enak, kot če bi iskali minimalno varianco brez danega začetnega pogoja -- ni pomembno, kakšne $n_1, n_2, \ldots, n_K$ izberemo. Stroški raziskave $C$ bodo vedno enaki, ker je cena $C_1$ vedno enaka.
\\

\noindent
\textsc{Primer b}
\\
Sedaj se lahko stroški opažanja spreminjajo od stratuma do strauma. Če je $n_k$ število enot iz $k-$tega stratuma, ki so zajete v vzorec, naj bodo stroški raziskave enaki:
$$ C = C_0 + \sum_{k = 1}^{K}n_k C_k. $$
Pri danih sredstvih za raziskavo v višini $C$ poiščemo tiste velikosti podvzorcev, pri katerih je varianca cenilke populacijskega povprečja minimalna.
\\
Rešujemo podoben primer kot prej, le da sedaj nimamo več fiksne $C_1$, ampak se spreminja z vsakim stratumom. 
Z enakim razmislekom kot prej sestavimo Lagrangeovo funkcijo:
$$ F(n_1, n_2, \ldots, n_K, \lambda) = f(n_1, n_2, \ldots, n_K) + \lambda (C_0 + \sum_{k = 1}^{K} n_k C_k - C). $$
Zapišemo parcialne odvode funkcije $F$ in jih enačimo z $0$.
$$ \frac{ \partial F}{\partial n_i} = - \frac{ \sigma_i^2}{n_i^2} \  W_i^2 + \lambda \ C_i = 0 \ \ \ \text{za} \ i = 1, 2, \ldots, K.  $$
Izrazimo $n_i$ in dobimo sistem enačb
\begin{align}\label{en:nkb}
n_i = \frac{W_i \sigma_i}{ \sqrt{ \lambda \ C_i}}
\end{align}
Da določimo $\lambda$, naredimo vsoto enačbe (\ref{en:nkb}) po $k$, $k = 1, 2, \ldots, K.$
$$ n = \frac{1}{\sqrt{\lambda}} \sum_{k = 1}^{K} \frac{W_k \sigma_k}{\sqrt{C_k}} $$
\begin{align}\label{en:lambdab}
\Rightarrow \sqrt{\lambda} = \frac{1}{n} \sum_{k = 1}^{K} \frac{W_k \sigma_k}{\sqrt{C_k}}
\end{align}
Če združimo (\ref{en:nkb}) in (\ref{en:lambdab}), dobimo:

$$ n_i = \frac{n}{\sqrt{C_i}} \ \frac{W_i \sigma_i}{\sum_{k = 1}^{K} \frac{W_k \sigma_k}{\sqrt{C_k}}}. $$
\\

\noindent
\textsc{Primer c}
\\
Naj se stroški raziskave izražajo na enak način kot v prejšnji točki, predpisano pa imamo natančnost raziskave, torej varianco cenilke. Poiščemo tiste vrednosti podvzorcev, pri katerih bodo stroški najmanjši.
\\
Želimo torej minimizirati stroške pri pogoju, da je varianca minimalna, tj. $ \text{var}()\overline{X}) = \sum_{k = 1}^{K} \frac{W_k^2 \sigma_k^2}{n_k}. $
\\
Rešujemo torej vezani ekstrem za funkcijo:
$$ f (n_1, n_2, \ldots, n_K) = C_0 + \sum_{k = 1}^{K} n_k C_k $$
$$ \text{z vezjo} \ \sum_{k = 1}^{K} \frac{W_k^2 \sigma_k^2}{n_k} - \text{var}(\overline{X}). $$
Sestavimo Lagrangeovo funkcijo:
$$ F (n_1, n_2, \ldots, n_K, \lambda) = C_0 + \sum_{k = 1}^{K} n_k C_k  + \lambda \left( \sum_{k = 1}^{K} \frac{W_k^2 \sigma_k^2}{n_k} - \text{var}(\overline{X}) \right)$$
Zapišemo parcialne odvode funkcije $F$ in jih enačimo z $0$.
$$ \frac{ \partial F}{\partial n_i} = C_i - \lambda \ \frac{W_i^2 \sigma_i^2}{n_i^2} = 0 \ \ \ \text{za} \ i = 1, 2, \ldots, K.  $$
Izrazimo $n_i$ in dobimo sistem enačb
\begin{align}\label{en:nkc}
n_i = \sqrt{\frac{\lambda}{C_i}} \ W_i \sigma_i
\end{align}
Da določimo $\lambda$, naredimo vsoto enačbe (\ref{en:nkc}) po $k$, $k = 1, 2, \ldots, K.$
$$ n = \sqrt{\lambda} \ \sum_{k = 1}^{K} \frac{W_k \sigma_k}{\sqrt{C_k}} $$
\begin{align}\label{en:lambdac}
\Rightarrow \sqrt{\lambda} = \frac{n}{\sum_{k = 1}^{K} \frac{W_k \sigma_k}{\sqrt{C_k}}}
\end{align}
Če združimo (\ref{en:nkc}) in (\ref{en:lambdac}), dobimo:

$$ n_i = \frac{n}{\sqrt{C_i}} \ \frac{W_i \sigma_i}{\sum_{k = 1}^{K} \frac{W_k \sigma_k}{\sqrt{C_k}}}. $$
\\

%%%%%%%%%%%%%%%%%%%%%%%%%%%%%%%%%%%%%%%%%%%%%%%%%%%%%%%%%%%%%%

\noindent
\textsc{\large{3. NALOGA}}
\\
\\
Opazimo $n$ neodvisnih realizacij zvezne porazdelitve z gostoto:
\begin{displaymath}
    f(x \ | \ \sigma) = \left\{ \begin{array}{ll}
     \frac{\Gamma(2 \alpha)}{(\Gamma (\alpha))^2} \left[ x(1 - x) \right]^{\alpha - 1} & \textrm{; \ $0 < x < 1$}\\
     0 & \textrm{; \ sicer,} \\
    \end{array} \right. 
\end{displaymath}
kjer je $\alpha > 0$ neznam parameter. Če je $X$ slučajna spremenljivka s to gostoto, se da izračunati:
$$ \text{E}(X) = \frac{1}{2} \ , \ \ \ \ \ \text{var}(X) = \frac{1}{4(2 \alpha + 1)}. $$
\\

\noindent
\textsc{Primer a}
\\
Določimo obliko porazdelitve v odvisnosti od $\alpha$.
\\

\noindent
\textsc{Primer b}
\\
Ocenimo $\alpha$ po metodi momentov.
\\
Iz podane pričakovane vrednosti in variance lahko izračunamo vrednosti prvega in drugega momenta.
$$ \text{E}(X) = \frac{1}{n} \sum_{i = 1}^{n} X_i = \frac{1}{3}, $$
$$ \text{E}(X^2) = \frac{1}{n} \sum_{i = 1}^{n} X_i^2 = \text{var}(X) + \text{E}(X)^2 = \frac{1}{4(2 \alpha + 1)} + \frac{1}{4}. $$

Iz enačbe drugega momenta izrazimo $\alpha$.
\begin{align*}
    \frac{1}{4(2 \alpha + 1)} + \frac{1}{4} &= \frac{1}{n} \sum_{i = 1}^{n} X_i^2 
    \\
    \frac{1}{4(2 \alpha + 1)} &= \frac{1}{n} \sum_{i = 1}^{n} X_i^2 - \frac{1}{4} 
    \\
    \frac{1}{2 \alpha + 1} &= \frac{4}{n} \sum_{i = 1}^{n} X_i^2 - 1 
    \\
    2 \alpha + 1 &= \frac{1}{\frac{4}{n} \sum_{i = 1}^{n} X_i^2 - 1}
    \\
    \Rightarrow \alpha &= \frac{1}{2} \left( \frac{1}{\frac{4}{n} \sum_{i = 1}^{n} X_i^2 - 1} - 1 \right).
\end{align*}

\noindent
\textsc{Primer c}
\\
Poiščemo enačbo, ki določa cenilko po metodi največjega verjetja. Pogledamo, kdaj ta cenilka obstaja.
\\

\noindent
\textsc{Primer d}
\\
Poiščemo asimptotično varianco cenilke po metodi največjega verjetja.
\\
 
%%%%%%%%%%%%%%%%%%%%%%%%%%%%%%%%%%%%%%%%%%%%%%%%%%%%%%%%%%%%%%

\noindent
\textsc{\large{4. NALOGA}}

%%%%%%%%%%%%%%%%%%%%%%%%%%%%%%%%%%%%%%%%%%%%%%%%%%%%%%%%%%%%%%

\noindent
\textsc{\large{5. NALOGA}}
\\
\\
$X$ in $Y$ sta slučajni spremenljivki, za kateri velja:
\begin{itemize} 
    \item $\text{E}(X) = \mu_x$,
    \item $\text{E}(Y) = \mu_y$,
    \item $\text{var}(X) = \sigma_x^2$,
    \item $\text{var}(Y) = \sigma_y^2$,
    \item $\text{cov}(X,Y) = \sigma_{x,y}$.
\end{itemize}
Opazimo $X$ in želimo napovedati $Y$. 
\\

\noindent
\textsc{Primer a}
\\
Poiščemo napoved, ki je oblike $\hat{Y} = \alpha + \beta \cdot X$, kjer $\alpha$ in $\beta$ izberemo tako, da je srednja kvadratična napaka $\text{E} \left[ \left( Y - \hat{Y} \right) ^2 \right]$ minimalna.
\\
Uporabimo namig 
\begin{align} 
    \text{E} \left[ \left( Y - \hat{Y} \right) ^2 \right] = \left[ \text{E}(Y) - \text{E}(\hat{Y}) \right]^2 + \text{var}(Y - \hat{Y}).
\end{align}
Ker sta oba člena desne strani enačbe večja od $0$, je dovolj, da poiščemo vrednosti $\alpha$ in $\beta$, ki minimizirata ta dva člena -- potem bo najmanjša možna tudi njuna vsota.
\\
Poglejmo si najprej prvi člen enačbe. Vrednost enačbe

\begin{align*}
    \left[ \text{E}(Y) - \text{E}(\hat{Y}) \right]^2 = \left[ \mu_y - \alpha - \beta \mu_x \right] ^2 
\end{align*}
bo najmanjša, ko bo $ \mu_y - \alpha - \beta \cdot \mu_x = 0$. To pa bo res natanko takrat, ko bo $\alpha = \mu_y - \beta \mu_x$. 
\\
Poglejmo si še drugi člen enačbe. Vidimo, da je 

\begin{align*} 
\text{var}(Y-\hat{Y}) &= \text{var}(Y - \alpha - \beta X) = \text{var}(Y - \beta X) = \text{var}(Y) - 2 \beta \text{cov}(X, Y) + \beta^2 \text{var}(X)
\\
&= \sigma_y^2 - 2 \beta \sigma_{x,y} + \beta^2 \sigma_x^2 
\end{align*}
funkcija spremenljivke $\beta$. Za izračun minimuma enačbo najprej odvajamo po $\beta$ in enačimo z $0$.

\begin{align*}
    \frac{\partial}{\partial \beta} (\text{var}(Y - \hat{Y})) = - 2 \sigma_{x,y} + 2 \beta \sigma_x^2 = 0 
\end{align*}
To bo res, ko bo $ \beta = \frac{\sigma_{x,y}}{\sigma_x^2} $.
\\
Vrednosti $\alpha$ in $\beta$, pri katerih bo izraz $\text{E} \left[ \left( Y - \hat{Y}) \right) ^2 \right]$ minimalen, sta
$$ \alpha = \mu_y - \mu_x \frac{\sigma_{x,y}}{\sigma_x^2} \ \ \ \ \ \text{in} \ \ \ \ \ \beta = \frac{\sigma_{x,y}}{\sigma_x^2} $$

\noindent
\textsc{Primer b}
\\
Pokažemo, da se pri tako izbranih koeficientih \textit{determinacijski koeficient} (kvadrat korelacijskega koeficienta) izraža v obliki
$$ r_{x,y}^2 = 1 - \frac{\text{var}(Y - \hat{Y})}{\text{var}(Y)}. $$
\\
Spomnimo se najprej formule za \textit{korelacijski koeficient}: 
$$ \text{corr}(X,Y) = \frac{\text{cov}(X,Y)}{\sqrt{\text{var}(X) \text{var}(Y)}} = r_{x,y}. $$
Ker je determinacijski koeficient enak kvadratu korelacije, je enak
$$ r_{x,y}^2 = \frac{\text{cov}(X,Y)^2}{\text{var}(X) \text{var}(Y)}. $$
Dobimo enakost
$$ 1 - \frac{\text{var}(Y - \hat{Y})}{\text{var}(Y)} = \frac{\text{cov}(X,Y)^2}{\text{var}(X) \text{var}(Y)} $$
in če upoštevamo še vrednosti $\alpha$ in $\beta$ iz prvega dela naloge:
\begin{align*} 
    1 - \frac{\text{var}(Y - \hat{Y})}{\text{var}(Y)} &= \frac{\text{var}(Y) - \text{var}(Y - \hat{Y})}{\text{var}(Y)} = \frac{\sigma_y^2 - \sigma_y^2 + \frac{\sigma_{x,y}^2}{\sigma_x^2}}{\sigma_y^2} 
    \\
    &= \frac{\sigma_{x,y}^2}{\sigma_x^2 \sigma_y^2} = \frac{\text{cov}(X,Y)^2}{\text{var}(X) \text{var}(Y)} = r_{x,y}^2. 
\end{align*}
Pokazali smo, da je determinacijski koeficient res enak $1 - \frac{\text{var}(Y - \hat{Y})}{\text{var}(Y)}$.
%%%%%%%%%%%%%%%%%%%%%%%%%%%%%%%%%%%%%%%%%%%%%%%%%%%%%%%%%%%%%%
\end{document}
