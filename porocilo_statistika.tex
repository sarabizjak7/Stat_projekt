\documentclass[a4paper,12pt]{article}

\usepackage[slovene]{babel}
\usepackage[utf8]{inputenc}
\usepackage[T1]{fontenc}
\usepackage{lmodern}
\usepackage{amsmath}
\usepackage{amssymb}
\usepackage{amsthm}
\usepackage{amsfonts}
\usepackage{mathtools}
\usepackage{enumitem}



\begin{document}

\begin{titlepage}
    \centering
    \vfill
    \vfill
    \textbf{\large{POROČILO SEMINARSKE NALOGE}}
    \vfill
    \textbf{\Huge{STATISTIKA}}
    \vfill\vfill\vfill\vfill\vfill
    \textsc{\Large{Sara Bizjak}}
    \vfill\vfill
    \textsc{\large{Univerza v Ljubljani}}
    
    \textsc{\large{Fakulteta za matematiko in fiziko}}
    
    \textsc{\large{Oddelek za matematiko}}
    \vfill\vfill
        
    \large{JULIJ 2020}
    
\end{titlepage}

%%%

\newpage

%%%%%%%%%%%%%%%%%%%%%%%%%%%%%%%%%%%%%%%%%%%%%%%%%%%%%%%%%%%%%%

\noindent
\textsc{\large{1. NALOGA}}
\\
\\
Podatki so vzeti iz datoteke \texttt{Kibergard}, kjer se nahajajo informacije o 43.886 družinah, ki stanujejo v mestu Kibergard. Za vsako družino so zabeleženi naslednji podatki:
\begin{itemize}
    \item Tip družine (od 1 do 3)
    \item Število članov družine
    \item Število otrok v družini
    \item Skupni dohodek družine
    \item Mestna četrt, v kateri stanuje družina (od 1 do 4)
    \item Stopnja izobrazbe in vodje gospodinjstva (od 31 do 46: opisi v datoteki z navodili)
\end{itemize}

\noindent
Nalogo sem reševala s pomočjo programa R. Koda, uporabljena za generiranje enostavnih slučajnih vzorcev in izračune, je dostopna v priloženi datoteki \textit{naloga1.R}. 
\\

%%%%%%%%%%%%%%%%%%%%%%%%%%%%%%%%

\noindent
\textsc{Primer a}
\\
Vzamemo enostavni slučajni vzorec 200 družin in na njegovi podlagi ocenimo delež družin v Kibergardu, v katerih vodja gospodinjstva nima srednješolske izobrazbe (niti poklicne niti splošne mature).
Opisan delež znaša $p = 0.195$.
\\

%%%%%%%%%%%%%%%%%%%%%%%%%%%%%%%%

\noindent
\textsc{Primer b}
\\
Ocenimo standardno napako in postavimo 95\% interval zaupanja.

\noindent
Standardno napako za delež izračunamo po formuli
$$ \hat{se}(p) = \sqrt{ \frac{p \cdot (1-p)}{n - 1} \cdot \left(1 - \frac{n}{N} \right)}, $$
kjer so $p = 0.195$, \ $n = 200$, \ $N = 43.886$. 
\\
Dobimo rezultat $\hat{se}(p) = 0.02802185$.
\\
Interval zaupanja je enak: $[0.1400782, 0.2499218]$.
\\

%%%%%%%%%%%%%%%%%%%%%%%%%%%%%%%%

\noindent
\textsc{Primer c}
\\
Vzorčni delež in ocenjeno standardno napako primerjamo s populacijskim deležem in pravo standardno napako. 
\\
\begin{itemize}
\item Vzorčni delež: $0.195$ 
\item Populacijski delež: $0.2115025$ 
\item Razlika obeh deležev: $0.01650253$
\item Ocenjena standardna napaka (iz vzorca): $0,02802185$
\item Prava standardna napaka (iz celotne populacije):  $0.02888282$
\item Razlika med ocenjeno in pravo standardno napako: $0.0008609634$
\end{itemize}
Ker velja $0.2115025 \in [0.1400782, 0.2499218]$, interval zaupanja pokrije populacijski delež.
\\

%%%%%%%%%%%%%%%%%%%%%%%%%%%%%%%%

\noindent
\textsc{Primer d}
\\
GRAF 
\\
INTERVALI ZAUPANJA + KOLIKO JIH POKRIJE POPULACIJSKI DELEŽ
\\

%%%%%%%%%%%%%%%%%%%%%%%%%%%%%%%%

\noindent
\textsc{Primer e}
\\
Standardni odklon vzorčnih deležev za $100$ prej dobljenih vzorcev je enak $0.02881085$. Prava standardna napaka za vzorec velikosti $200$ pa je $0.02888282$.
Razlikujeta se za $7.196778 \cdot 10^{-5}$.
\\

%%%%%%%%%%%%%%%%%%%%%%%%%%%%%%%%

\noindent
\textsc{Primer f}
\\
GRAF 
\\
INTERVALI ZAUPANJA + KOLIKO JIH POKRIJE POPULACIJSKI DELEŽ

%%%%%%%%%%%%%%%%%%%%%%%%%%%%%%%%%%%%%%%%%%%%%%%%%%%%%%%%%%%%%%

\noindent
\textsc{\large{2. NALOGA}}

%%%%%%%%%%%%%%%%%%%%%%%%%%%%%%%%%%%%%%%%%%%%%%%%%%%%%%%%%%%%%%

\noindent
\textsc{\large{3. NALOGA}}

%%%%%%%%%%%%%%%%%%%%%%%%%%%%%%%%%%%%%%%%%%%%%%%%%%%%%%%%%%%%%%

\noindent
\textsc{\large{4. NALOGA}}

%%%%%%%%%%%%%%%%%%%%%%%%%%%%%%%%%%%%%%%%%%%%%%%%%%%%%%%%%%%%%%

\noindent
\textsc{\large{5. NALOGA}}

%%%%%%%%%%%%%%%%%%%%%%%%%%%%%%%%%%%%%%%%%%%%%%%%%%%%%%%%%%%%%%

\end{document}
